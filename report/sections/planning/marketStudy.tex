\section{Estudo do mercado}

Ao longo do processo de decisão da área de negócio sobre a qual o grupo de trabalho iria incidir, surgiram várias ideias, que foram descartadas porque existiam várias alternativas que eram muito boas, nas quais o acréscimo de funcionalidades não seria a melhor opção porque não iria abranger o público-alvo que era objetivo.

Com isto, idealizou-se o projeto de partilha de despesas, mas ao se verificarem várias fontes de dados, as possibilidades com as \textit{keywords} "split" e "bill", retornavam muitos resultados, contudo, em nenhum destes se verificaram possibilidades que tinham como público-alvo os estudantes, aparecendo todos como genéricos, o que como é óbvio, não se centraliza nas necessidades que estes têm. Com base nisto, não se encontraram concorrentes diretos a esta área de negócio.

A falta de concorrência não permite dar o conhecimento sobre os requisitos operacionais que um projeto desta dimensão terá, porque através destes, é possível ter conhecimentos sobre:
\begin{itemize}
\item
Avaliação do crescimento do mercado
\item
Analisar o comportamento e as tendências do setor.
\item
Reconhecer estratégias vencedoras.
\item
Identificar os concorrentes e o seu valor.
\end{itemize}

Apesar desta falta de resultados, a área de negócio na qual o projeto se incide, terá uma boa aceitação pelos utilizadores, uma vez que o grupo sabe quais as necessidades dos estudantes, porque todos os elementos do grupo são estudantes e sabem que esta aplicação vai ser bastante útil.

Uma estratégia levada em consideração pelo grupo, é não introduzir qualquer custo ao utilizador. Com isto, espera-se que estes experimentem a aplicação apenas por curiosidade. Com base nesta experimentação, é necessário mostrar ao utilizador as vantagens induzidas por esta perante as suas concorrentes. Sabendo que esta tem como público-alvo os estudantes, é preciso que esta seja simplista, de modo a cativar a atenção dos mesmos.

O comportamento natural dos estudantes é seguir as tendências, e se o utilizador que experimentar gostar da aplicação, então o número de estudantes a usufruir da mesma irá crescer. Este crescimento irá induzir uma maior divulgação que permitirá aumentar a tendência da sua utilização.

Durante este estudo de mercado, verificaram-se dois cenários, sendo que o primeiro refere-se às aplicações concorrentes, nas quais existe uma lógica de negócio disponibilizada para qualquer pessoa que utiliza a aplicação, e o segundo cenário refere-se às alternativas, que passam por técnicas primitivas, tal como, documentos de texto. Nas duas subsecções seguintes apresentam as diferentes opções com mais detalhe.

\subsection{Aplicações Concorrentes}

Todas as aplicações, mesmo que tenham o mesmo tema, são sempre realizadas de forma diferente. Variam essencialmente, nas suas funcionalidades, nos seus algoritmos, na sua simplicidade, e obviamente na interface gráfica.

No contexto do projeto em desenvolvimento, tal como já foi referido, a maioria das aplicações existentes tentam ser genéricas para poderem obter um maior número de utilizadores, e com isto reparou-se que estas falham em alguns aspetos, tal como a abstração que se tem perante um tema. A título de exemplo, para um casal que habita numa casa própria, uma despesa de eletricidade pode ter uma semântica, mas para um estudante que partilha a casa com mais colegas terá outro significado.

O aumento da abstração dos conceitos é o principal problema que as aplicações concorrentes têm, porque para se conseguir abranger todos os tipos de grupos, é necessário ter conhecimento de todos os estilos de pessoas que irão utilizar a aplicação, e isso será muito complicado.

Deste modo, serão apresentadas algumas aplicações concorrentes, que apesar de não serem totalmente orientadas para os estudantes, permitem que possam ser usadas pelos mesmos.\\

\textbf{\\Split Wise}
\begin{enumerate}
\item Permite criar grupos de habitação e de viagens.
\item Em qualquer dos grupos, é possível criar despesas que são associadas a amigos. Estas são divididades forma equitativa por todos, contudo, pode-se mudar para pôr exatamente as quantias ou dividir por percentagens.
\item Triangular as despesas
\item Verificar gráficos
\item Realizar comentários nas despesas.
\item Login com o google+
\item Aplicação móvel para android e ios
\item Envio de lembretes de pagamento aos amigos.
\item Definir prazos de pagamento.
\end{enumerate}

\textbf{\\Split a Bill}
\begin{enumerate}
\item Criar nova despesa fatura e criar várias despesas para essa fatura.
\item Aplicação móvel para ios.
\item Tem conversações para a fatura.
\item Partilhar a despesa com vários amigos mas apenas de forma equitativa.
\item Envio de lembretes de pagamento aos amigos.
\end{enumerate}

\textbf{\\We Split}
\begin{enumerate}
\item Criar um grupo
\item Dentro do grupo criar fatura ou despesa
\item Dividir despesa de forma automática (equitativa) ou manual.
\item Adicionar notas à despesa.
\item Aplicação móvel para android.
\end{enumerate}

\textbf{\\Bill Pin}
\begin{enumerate}
\item Aplicação móvel para ios e android.
\item Apenas cria despesas sem agrupar por grupos ou faturas.
\item Adicionar despesa e associar amigos.
\item Gerar balanço de contas para email.
\end{enumerate}

\textbf{\\Bills Up}
\begin{enumerate}
\item Aplicação móvel para ios e android e windows phone.
\item Criar grupos e associar amigos.
\item Triangular.
\item Criar Despesa e associar amigos.
\item Importar e Exportar dados.
\end{enumerate}

\section{Alternativas}

De seguida serão apresentadas algumas soluções que apesar de não serem exatamente iguais à que se está a representar neste documento, apresentam algumas características comuns. Assim, serão apresentadas algumas vantagens e desvantagens.

\subsection{Documento de Texto}

Esta forma de divisão de contas, é o mais básico que pode ser realizado, contudo é muito desorganizado e certamente centralizado. Assim, o utilizador regista de forma textual todas as suas dívidas, assim como todas as quantias que têm a receber. Esta até pode ser uma boa solução, se a cardinalidade dos registos for baixa, contudo, quando o volume de dados aumento, a complexidade na gestão da informação também será muito maior.

Além disto, podem haver muitos problemas com valores inconsistentes, porque como não há sincronização entre os utilizadores, estes podem colocar valores distintos nos seus registos.

Importa ainda salientar, que com esta abordagem, toda a consulta e gestão de dívida é muito mais complicada, porque o processo não é automático.

Este formato é completamente desaconselhado, mas acredita-se que ainda há muitos grupos de pessoas que preferem utilizar este tipo de organização.

\subsection{Folhas de Cálculo}

Contrariamente à aboragem apresentada na subsecção anterior, a utilização de folhas de cálculo (ficheiros Excel, etc) é uma melhor alternativa. Mas, o problema da sincronização continua presente. É certo que com as tecnologias habituais, já é possível sincronizar os mesmos, porém o processo de partilha era mais complicado, uma vez que seria necessário partilhar o documento apenas com as pessoas que lhe dizem respeito, tendo no limite, um ficheiro para cada evento.

Além disso, a consulta de valores necessários para a tomada de decisão é mais delicada, sendo necessário analisar todos os ficheiros e agregar apenas os valores oportunos.

A vantagem da utilização desta abordagem deve-se ao facto de que cada utilizador cria o ficheiro de acordo com as suas necessidades, ajustando-o aos seus gostos. Mas, é óbvio que este requer algum tempo para ser construído, obrigando a que o utilizador tenha algum conhecimento para que as possa ajustar às suas necessidades. Além disso, é certo que estas estarão mais suscetíveis a erros.