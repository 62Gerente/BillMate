\section{Alternativas}

De seguida serão apresentadas algumas soluções que apesar de não serem exatamente iguais à que se está a representar neste documento, apresentam algumas características comuns. Assim, serão apresentadas algumas vantagens e desvantagens.

\subsection{Documento de Texto}

Esta forma de divisão de contas, é o mais básico que pode ser realizado, contudo é muito desorganizado e certamente centralizado. Assim, o utilizador regista de forma textual todas as suas dívidas, assim como todas as quantias que têm a receber. Esta até pode ser uma boa solução, se a cardinalidade dos registos for baixa, contudo, quando o volume de dados aumento, a complexidade na gestão da informação também será muito maior.

Além disto, podem haver muitos problemas com valores inconsistentes, porque como não há sincronização entre os utilizadores, estes podem colocar valores distintos nos seus registos.

Importa ainda salientar, que com esta abordagem, toda a consulta e gestão de dívida é muito mais complicada, porque o processo não é automático.

Este formato é completamente desaconselhado, mas acredita-se que ainda há muitos grupos de pessoas que preferem utilizar este tipo de organização.

\subsection{Folhas de Cálculo}

Contrariamente à aboragem apresentada na subsecção anterior, a utilização de folhas de cálculo (ficheiros Excel, etc) é uma melhor alternativa. Mas, o problema da sincronização continua presente. É certo que com as tecnologias habituais, já é possível sincronizar os mesmos, porém o processo de partilha era mais complicado, uma vez que seria necessário partilhar o documento apenas com as pessoas que lhe dizem respeito, tendo no limite, um ficheiro para cada evento.

Além disso, a consulta de valores necessários para a tomada de decisão é mais delicada, sendo necessário analisar todos os ficheiros e agregar apenas os valores oportunos.

A vantagem da utilização desta abordagem deve-se ao facto de que cada utilizador cria o ficheiro de acordo com as suas necessidades, ajustando-o aos seus gostos. Mas, é óbvio que este requer algum tempo para ser construído, obrigando a que o utilizador tenha algum conhecimento para que as possa ajustar às suas necessidades. Além disso, é certo que estas estarão mais suscetíveis a erros.