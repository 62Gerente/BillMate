\section{Plataformas}

Durante o planeamento do projeto ficou delineado que o objetivo será desenvolver uma aplicação que estará disponível em suporte web e em suporte mobile.

Sabe-se que o crescimento das pessoas que navegam diariamente na internet através de um computador está a estagnar, contudo, o crescimento na navegação através de dispositivos móveis está a crescer como nunca. Principalmente desde o ano de 2010, o paradigma tem-se alterado de web para mobile. Com isto, não se quer dizer que o desenvolvimento \textit{web} já esteja estagnado, pelo contrário, mas cada vez mais é necessário desenvolver sites \textit{responsive}, para que estes se ajustem às pequenas telas dos dispositivos móveis.

Para a escolha foram necessários alguns fatores de decisão, tais como:

\begin{description}
  \item[Funcionalidades do dispositivo] \hfill \\
  Apesar das aplicações \textit{web} poderem usufruir das funcionalidades dos dispositivos, não o conseguem da mesma forma que as aplicações nativas, uma vez que estas são realizadas especificamente para o dispositivo que a está a executar.
  \item[Funcionamento offline] \hfill \\
  Apesar da funcionalidade \textit{offline} estar presente como um objetivo futuro do grupo, é certo que para se conseguir isso, é mais fácil com as aplicações nativas, porque além do \textit{browser} possuír uma cache, ainda é limitada comparativamente ao nativo.
  \item[Descoberta] \hfill \\
  Como se pretende que esta seja divulgada, é certo que as aplicações \textit{web} são encontradas de forma mais fácil, porque esta está constantemente na \textit{web}, e quando se pesquisa uma determinada \textit{keyword}, é muito provável que apareça nos resultados da pesquisa.
  \item[Velocidade] \hfill \\
  A velocidade é um fator crucial para esta aplicação, e neste caso as aplicações nativas são mais rápidas por terem acesso direto ao sistema operativo e além disso são programadas na linguagem nativa do dispositivo.
  \item[Instalação] \hfill \\
  Alguns utilizadores não gostam de ter de instalar aplicações para usufruir das funcionalidades, deste modo a escolha do grupo recai sobre a construção da aplicação \textit{web}, contudo, tal como já foi referido anteriormente atras, com esta não é possível obter o mesmo nível da \textit{User Experience} que o obtido com as aplicações nativas.
  \item[Manutenção] \hfill \\
  Devido a atualizações, é necessário que a aplicação sofra manutenção, deste modo, opta-se pelas aplicações \textit{web} pois podem ser atualizados com a frequência necessária como se fosse uma página da internet. As aplicações nativas têm mais problemas neste sentido, porque é necessário que o utilizador esteja atento às atualizações que vão surgindo.
  \item[Cross-Platform] \hfill \\
  Como é óbvio, a aplicação \textit{web} apenas necessita de um \textit{browser} para poder executar, enquanto que as aplicações nativas são feitas exclusivamente para um sistema operativo.
  \item[Interface ao utilizador] \hfill \\
  Como se pretende aumentar a \textit{User Experience}, a aplicação nativa consegue tirar mais proveito dos componentes do dispositivo. Mas, isto não significa que não se consiga bons resultados com uma aplicação \textit{web}, mas como foi dito, a usabilidade não será a mesma
\end{description}

Uma vez que todos os fatores retratados anteriormente foram importantes para a escolha do tipo da aplicação, é fácil de perceber que neste caso não há um tipo que contenha todas estas características. Assim, para o grupo conseguir atingir os objetivos para esta aplicação, optou por desenvolver uma aplicação com um conteúdo \textit{web} e um conteúdo \textit{mobile}, sendo que numa primeira instância apenas terá suporte em \textit{Android e iOS}.