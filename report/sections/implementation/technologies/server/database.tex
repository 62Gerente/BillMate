\subsection{Base de dados - PostgreSQL} 

A escolha de um Sistema de Gestão de Bases de Dados (\emph{DBMS}) torna-se fulcral a partir do momento em que se pretende obter escalabilidade, confiança e técnicas de replicação por este fornecidas. Desta forma, é preciso pesquisar quais oferecem estas garantias e quais as que têm um maior suporte, ou até mesmo uma comunidade de suporte, capaz de responder às mais variadas dúvidas sobre o produto em questão.

O open-source costuma, por si só, ter muitos adeptos e, quando é algo que é usado por grandes companhias, torna-se uma das prioridades na hora da escolha. O mesmo se sucedeu no que toca ao \emph{BillMate}, ou seja, acabou por se escolher um \emph{DBMS open-source}, com uma excelente reputação, quer a nível pessoal, quer a nível empresarial: \emph{PostgreSQL}.

Resumindo, os factores que influenciaram a escolha do \emph{PostgreSQL} foram:
\begin{itemize}
  \item Solução open-source, o que contribui para um estudo fácil do código, se necessário;
  \item Comunidade que contribui para uma curva de aprendizagem não tão elevada, devido à resolução de problemas;
  \item \emph{Cross-platform} é uma mais valia para se adaptar a qualquer infraestrutura e sistema que a compõe;
  \item Escalabidade, com o objectivo de atingir um desempenho suficiente para atender múltiplos clientes.
\end{itemize}