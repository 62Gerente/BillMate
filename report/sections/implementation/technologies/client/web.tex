\subsection{Web} 

	\textbf{\large{HTML5}}\\
	\label{ssub:html5}
	
	O \emph{HTML5}, na sua lenta chegada, iniciou uma simplificação daquilo que se conhecia como \emph{HTML}. Algumas \emph{tags} foram removidas (algumas delas cujo efeito seria o mesmo que aplicar um estilo) e novas foram introduzidas, tornando o código cada vez mais legível e facilmente editável, assim que for necessário, devido aos seus nomes simples e cada vez mais próximos da linguagem do dia-a-dia. A seleção desta tecnologia para o cliente foi vista como uma mais valia devido à simplicidade e furor, pelo qual é respomsável, em volta dos \emph{developers}.\\

	\textbf{\large{CSS3}}\\
	\label{ssub:css3}
	
	O \emph{CSS3} mudou completamente a imagem dos estilos, à medida que foi aparecendo. Várias animações deixam de ser apenas do domínio do \emph{JavaScript} e passam a ser totalmente personalizáveis naquilo que antes se chamaria apenas "estilos". Para um conforto e dinâmica da aparência do serviço, optou-se por adotar a versão 3 do famoso \emph{CSS}. Note, também, que o uso da \emph{framework} \emph{BootStrap 3} tem como base o \emph{CSS3}, na sua maioria.\\

	\textbf{\large{JavaScript}}\\
	\label{ssub:javascript}
	
	Já não sendo visto tanto como um atalho para animações ou alterações muito focadas no \emph{DOM}, esta linguagem traz uma flexibilidade de processamento do lado cliente, o que pode reduzir o processamento no lado servidor e, por essa grande vantagem, foi uma tecnologia escolhida e valorizada na implementação do cliente. Para uma maior flexibilidade, optou-se,também por usar \textit{jQuery}.\\

	\textbf{\large{Ajax}}\\
	\label{ssub:ajax}
	
	Não é uma linguagem! Cai no vasto leque de tecnologias existentes para a \emph{web}. Usa \emph{JavaScript} como base, visto que corre num \emph{browser} e tira partido do facto de não haver (ou haver pouca) transferência de \emph{HTML} por parte de cada pedido ao servidor. Foca-se no uso de \emph{JSON}, maioritariamente e, por isso, a inexistência do \emph{overload} dado pela transmissão de \emph{HTML} não prejudica, muito pelo contrário, a \emph{performance} do serviço.\\