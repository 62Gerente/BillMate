\chapter[Motivação]
{Motiva\c{c}\~ao}

Todas as pessoas necessitam de organizar os seus débitos e os seus ativos, porém, por lapso ou por falta de tempo, nem sempre se lembram de todas as suas obrigações. No caso dos estudantes universitários, isto ainda é um fator mais agravante, porque estes começam a sentir esta necessidade quando vão para a universidade, porque a maior parte destes, saem da casa dos pais nesta altura, o que implica que iniciam novas responsabilidades que até este momento não tinham.

Muitas são as histórias de falta de eletricidade, falta de gás, ou até mesmo de falta de internet por atraso no pagamento. Esta aplicação tende a responder a estas necessidades. Estes problemas são contornados devido aos alertas e notificações que são enviados ao utilizador para que este não se esqueça das suas obrigações.

A grande vantagem desta aplicação, deve-se ao facto de ser elaborada por estudantes, e destinar-se a estudantes, pois, quem está a fazer a aplicação sabe as necessidades para este contexto. Não há melhor cliente do que quem elabora o produto, porque sabe todos os seus requisitos. O feedback demonstrado pelo público-alvo tem sido bastante positivo, demonstrando bastante interesse pelo resultado final desta aplicação.

Esta aplicação é bastante ambiciosa, porque a ambição do grupo também é bastante relevante, uma vez que no final deste trabalho, pretende-se que esta aplicação não fique "apenas no papel", e que saia do contexto académico, ficando online e disponível para todas as pessoas.
