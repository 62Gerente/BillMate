\chapter[How to?]
{Howto}

\section{Equações}

\begin{equation}
ABC {\cal DEF} \alpha\beta\Gamma\Delta\sum^{abc}_{def}
\end{equation}


\section{Referenciar a bibliografia}

zhamming - \cite{hamming} e ainda zHu - \cite{zHu}

\section{Inserir Imagem}

\begin{figure}[H]
\centerline{\includegraphics[width=.5\textwidth]{images/dium}}
\caption{Legenda.}
\end{figure}

\pagebreak

Aqui está outra, mas inseri uma quebra de página para se conseguir separar as duas:

\begin{figure}[H]
\vskip2pt
\centerline{\includegraphics[width=.5\textwidth]{images/dium}}
\caption{Oscillograph for  memory address access operations,
showing 500 ps
address access time and superimposed signals
of address access in 1 kbit
memory plane.}
\end{figure}


\pagebreak

\section{Inserir Tabela}

\begin{table}[ht]
\caption{Small Table}
\centering
\begin{tabular}{cccc}
\hline
one&two&three&four\\
\hline
C&D&E&F\\
\hline
\end{tabular}
\end{table}


\section{Inserir um Exemplo}
\vskip6pt
\begin{example}[Titulo do Exemplo]
Apresentar um exemplo de qualquer coisa.
\end{example}


\section{2 imagens lado a lado}

\begin{figure}[H]
\sidebyside{
\includegraphics[width=.5\textwidth]{images/dium}
\caption{Imagem da Esquerda}
}
{
\includegraphics[width=.5\textwidth]{images/dium}
\caption{Imagem da Direita}
}
\end{figure}


\section{Inserir Tabela de outro estilo}

\begin{table}[ht]
\caption{Effects of the two types of $\alpha\beta\sum^A_B$ scaling proposed by Dennard \newline
and
co-workers$^{a,b}$}
\begin{tabular*}{\textwidth}{@{\extracolsep{\fill}}lcc}
\hline
Parameter& $\kappa$ Scaling & $\kappa$, $\lambda$ Scaling\cr
\hline
Dimension&$\kappa^{-1}$&$\lambda^{-1}$\cr
Voltage&$\kappa^{-1}$&$\kappa^{-1}$\cr
Currant&$\kappa^{-1}$&$\lambda/\kappa^{2}$\cr
Dopant Concentration&$\kappa$&$\lambda^2/\kappa$\cr
\hline
\end{tabular*}
\begin{tablenotes}
$^a$Refs.~19 and 20.

$^b\kappa, \lambda>1$.
\end{tablenotes}
\end{table}


\section{Inserir Tabela lado a lado}

Tabelas lado a lado

 \begin{table}[ht]
 \sidebyside{
\caption{Table Caption}
\begin{tabular}{cccc}
one&two&three&four\\
a &little&sample&table
\end{tabular}
}
 {
\caption{Table Caption}
\begin{tabular}{cccc}
A&B&C&D\\
a &second little& sample&table
\end{tabular}
}
 \end{table}


\section{Usar Verbatim}

\begin{verbatim}
 \begin{table}
 \sidebyside{\caption{Table Caption}\label{tab1}
 first table}
 {\caption{Table Caption}\label{tab2} second table}
 \end{table}
\end{verbatim}



\section{Colocar Snippets}

\insertcode{snippets/example.pl}{Nena would be proud.}


\section{Colocar Algoritmo}

\begin{algorithm}
{\bf state\_transition algorithm} $\{$
\        for each neuron $j\in\{0,1,\ldots,M-1\}$
\        $\{$
\            calculate the weighted sum $S_j$ using Eq. (6);
\            if ($S_j>t_j$)
\                    $\{$turn ON neuron; $Y_1=+1\}$
\            else if ($S_j<t_j$)
\                    $\{$turn OFF neuron; $Y_1=-1\}$
\            else
\                    $\{$no change in neuron state; $y_j$ remains %
unchanged;$\}$
\        $\}$
$\}$
\end{algorithm}

\section{Inserir Citação}

\begin{quote}
	This is a sample of extract or quotation.
\end{quote}


\section{Criar Listas}

Tipo 1 \newline\newline

\begin{enumerate}
\item
This is the first item in the numbered list.

\item
This is the second item in the numbered list.
This is the second item in the numbered list.
This is the second item in the numbered list.
\end{enumerate}

Tipo 2 \newline\newline

\begin{itemize}
\item
This is the first item in the itemized list.

\item
This is the first item in the itemized list.
This is the first item in the itemized list.
This is the first item in the itemized list.
\end{itemize}

Tipo 3 \newline\newline

\begin{itemize}
\item[]
This is the first item in the itemized list.

\item[]
This is the first item in the itemized list.
This is the first item in the itemized list.
This is the first item in the itemized list.
\end{itemize}


\section{Criar Problemas}

\begin{problems}
\prob
For Hooker's data, Problem 1.2, use the Box and Cox and Atkinson procedures to determine a appropriate transformation of PRES
in the regression of PRES on TEMP. find $\hat\lambda$, $\tilde\lambda$,
the score test, and the added variable plot for the score.
Summarize the results.

\prob
The following data were collected in a study of the effect of dissolved sulfur
on the surface tension of liquid copper (Baes and Killogg, 1953).

{\centering
\vskip6pt
\begin{tabular}{rlcc}
\hline
&&\multicolumn2c{$Y$= Decrease in Surface Tension}\\
\multicolumn2c{$x$ = Weight \% sulfur}
&\multicolumn2c{(dynes/cm), two Replicates}\\
\hline
0.&034&301&316\\
0.&093&430&422\\
0.&30&593&586\\
\hline
\end{tabular}
\vskip6pt
}


\subprob
Find the transformations of $X$ and $Y$ sot that in the transformed scale
the regression is linear.

\subprob
Assuming that $X$ is transformed to $\ln(X)$, which choice of $Y$ gives
better results,
$Y$ or $\ln(Y)$? (Sclove, 1972).

\sidebysidesubprob{In the case of $\alpha_1$?}{In the case of $\alpha_2$?}

\prob
Examine the Longley data, Problem 3.3, for applicability of assumptions of the
linear model.

\sidebysideprob{In the case of $\Gamma_1$?}{In the case of $\Gamma_2$?}

\end{problems}


\section{Criar Exercícios}


\begin{exercises}
\exer
For Hooker's data, Exercise 1.2, use the Box and Cox and Atkinson procedures to determine a appropriate transformation of PRES
in the regression of PRES on TEMP. find $\hat\lambda$, $\tilde\lambda$,
the score test, and the added variable plot for the score.
Summarize the results.

\exer
The following data were collected in a study of the effect of dissolved sulfur
on the surface tension of liquid copper (Baes and Killogg, 1953).

{\centering
\vskip6pt
\begin{tabular}{rlcc}
\hline
&&\multicolumn2c{$Y$= Decrease in Surface Tension}\\
\multicolumn2c{$x$ = Weight \% sulfur}
&\multicolumn2c{(dynes/cm), two Replicates}\\
\hline
0.&034&301&316\\
0.&093&430&422\\
0.&30&593&586\\
\hline
\end{tabular}
\vskip6pt
}


\subexer
Find the transformations of $X$ and $Y$ sot that in the transformed scale
the regression is linear.

\subexer
Assuming that $X$ is transformed to $\ln(X)$, which choice of $Y$ gives
better results,
$Y$ or $\ln(Y)$? (Sclove, 1972).

\sidebysidesubexer{In the case of $\Delta_1$?}{In the case of $\Delta_2$?}

\exer
Examine the Longley data, Problem 3.3, for applicability of assumptions of the
linear model.

\sidebysideexer{In the case of $\Gamma_1$?}{In the case of $\Gamma_2$?}

\end{exercises}


\section{Criar Apêndice com o título}

\chapappendix{Titulo do Apêndice}
Aqui mete-se a descrição. Nota: O NOME DA LABEL FICA DIFERENTE, REALÇANDO QUE É UM APÊNDICE. EXEMPLO: FICA FIGURA 4-A.1
\begin{equation}
\alpha\beta\Gamma\Delta
\end{equation}


\begin{figure}[H]
\caption{This is an appendix figure caption.}
\end{figure}

\begin{table}[ht]
\caption{This is an appendix table caption}
\centering
\let\hline\savehline
\begin{tabular}{@{\vrule height 11pt depth 4pt width0pt}|l|p{.65\textwidth}|c}
\hline
{\bf Date} & \multicolumn1{c|}{\bf Event} \\
\hline \hline
1867 & Maxwell speculated the existence of electromagnetic waves.\\
1887 & Hertz showed the existence of electromagnetic waves. \\
1890 & Branly developed technique for detecting radio waves. \\
1896 & Marconi demonstrated wireless telegraph. \\
1897 & Marconi patented wireless telegraph.  \\
1898 & Marconi awarded patent for tuned communication. \\
1898 & Wireless telegraphic connection between England and France established. \\
\hline
\end{tabular}
\end{table}



\section{Criar Índice Remissivo}

pedro leite \index{pedro leite}.
andre santos \index{andre santos}.
francisco neves \index{francisco neves}.
ricardo branco \index{ricardo branco}.
Universidade do Minho \index{Universidade do Minho}.
Departamento de Informática \index{Universidade do Minho!Departamento de Informática}.
Departamento de Produção e Sistemas \index{Universidade do Minho!Departamento de Produção e Sistemas}.
Engenharia Informática \index{Departamento de Informática!Engenharia Informática}.


\begin{verbatim}
	pedro leite \index{pedro leite}.
	andre santos \index{andre santos}.
	francisco neves \index{francisco neves}.
	ricardo branco \index{ricardo branco}.
	Universidade do Minho \index{Universidade do Minho}.
	Departamento de Informática \index{Universidade do Minho!Departamento de Informática}.
	Departamento de Produção e Sistemas \index{Universidade do Minho!Departamento de Produção e Sistemas}.
	Engenharia Informática \index{Departamento de Informática!Engenharia Informática}.
\end{verbatim}
