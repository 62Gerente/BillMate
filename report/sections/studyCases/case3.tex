\section{Caso 3 - Jantar}

O Pedro, o André, o Francisco e o Ricardo foram jantar a seguir às aulas, mas quem pagou toda a despesa foi o Pedro. Com o objetivo de controlar quem já pagou e quem ainda falta pagar, o Pedro criou um grupo e associou os amigos à mesma, e dividiu a despesa pelos seus amigos.

Os seus amigos recebem a notificação, procedem ao pagamento e confirmam o pagamento. De seguida o Pedro confirma que recebeu a quantia monetária. Mas, por lapso, o André esqueceu-se de pagar a despesa, e como o Pedro não definiu um limite de pagamento, o André não recebeu nenhuma notificação a lembrar da sua dívida.

O Pedro, ao analisar o seu histórico e e sua lista de débitos, verificou que tinha uma quantia ainda por receber. Assim, o Pedro relembra o André da sua dívida, e este por sua vez, procede ao pagamento ao Pedro e confirma o mesmo. Da mesma forma, o Pedro confirma que recebeu a quantia do André, e a despesa criada fica liquidada.

Uma vez que o Pedro tem todas as despesas liquidadas, ele gera um relatório detalhado para verificar as suas transações e analisar os seus gastos.