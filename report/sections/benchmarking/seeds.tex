\section{Povoação da base de dados}

Por forma a incluir alguma carga de registos no serviço, tais como utilizadores, círculos, despesas e pagamentos, optou-se por implementar um \emph{script} na linguagem \emph{PHP}.


A linha de pensamento seguida foi:
\begin{itemize}
	\item \textbf{Adicionar utilizadores no sistema} \\
		\indent A adição de utilizadores é necessária, visto que todo o resto do procedimento de geração de dados se torna dependente destes. Para obter a lista de utilizadores, recorreu-se à \emph{API} do \emph{www.randomuser.me}. Para registar utilizadores no sistema, usou-se o seguinte comando:
		\begin{verbatim}
			php seedRegisteredUser.php <quantity>
		\end{verbatim}

	\item \textbf{Adicionar círculos de utilizadores} \\
		\indent Tendo em conta que, no sistema, os círculos são os pontos de encontro de utilizadores, torna-se essencial gerar alguns de modo a que possam interagir através da adição de despesas e os respetivos pagamentos.
		\begin{verbatim}
			php seedCircle.php <quantity>
		\end{verbatim}

	\item \textbf{Adicionar despesas aos círculos} \\
		\indent Tendo em conta que, no sistema, os círculos são os pontos de encontro de utilizadores, torna-se essencial gerar alguns de modo a que possam interagir através da adição de despesas e os respetivos pagamentos.
		\begin{verbatim}
			php seedExpense.php <quantity_per_circle>
		\end{verbatim}

	\item \textbf{Adicionar pagamentos de despesas} \\
		\indent Para simular algum estado com algumas alterações, foram simulados pagamentos de despesa, de modo a verificar diferentes estados da aplicação.
		\begin{verbatim}
			php seedPayment.php
		\end{verbatim}
\end{itemize}

Note que os estados intermédios de cada comando são recolhidos do estado da base de dados e guardados localmente, no formato \emph{JSON}.

Desta forma, depois de inseridos os estados, pode dar-se início à rotina de recolha de dados sobre o comportamento do sistema face às cargas a que é sujeito, que será explicada de seguida.
