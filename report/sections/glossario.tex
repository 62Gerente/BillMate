\begin{glossary}

\term{Android}É um sistema operativo baseado no núcleo do Linux para dispositivos móveis, desenvolvido pela \textit{Open Handset Alliance} que é liderada pela \textit{Google}.

\term{Git}Git é um sistema de controlo de versões distribuído e um sistema de gestão de código fonte, com ênfase na velocidade. Começou por ser projetado e desenvolvido para o desenvolvimento do kernel Linux, mas foi adotado por muitos outros projetos, porque este permite que cada diretório de trabalho do Git seja um repositório com um histórico completo e habilidade total de acompanhamento das revisões, não dependente de acesso a uma rede ou a um servidor central.

\term{GitHub}É um serviço \textit{web hosting} para o desenvolvimento de projetos de software que usa o sistema de controlo de versões do Git. Este permite que todos trabalhem em um repositório único criando vários "ramos", que mais tarde serão unidos após a aceitação de todos os colaboradores desse repositório.

\term{Grails}\textit{Framework} para desenvolvimento de aplicações para web, utilizando a linguagem Groovy. Permite criar uma \textit{Framework} de alta produtividade em \textit{JAVA}. Segue o paradigma da programação por convenção que torna os detalhes de configuração transparentes para o programador.

\term{Groovy}Linguagem de programação orientada aos objetos, similar ao \textit{JAVA}, que se apresenta como alternativa. Apesar desta semelhança, possui características de \textit{Python} e \textit{Ruby}. É compilada em \textit{bytecode}, integrando-se facilmente com outras bibliotecas em \textit{JAVA}.

\term{IntelliJ}\textit{JAVA IDE} que pertence à \textit{JetBrains}, disponível com a edição de comunidade e uma edição comercial.

\term{Trello}É uma aplicação \textit{web-based} para a gestão de projetos que utiliza o paradigma conhecido por \textit{kanban}. Os projetos são representados por \textit{boards}, e cada uma tem várias listas, que podem ser entendidas como listas de tarefas. Cada lista contém vários \textit{cards}, em que cada uma corresponde a uma tarefa, e representam o fluxo que a lista de tarefas terá. Deste modo, os utilizadores podem ser associados aos \textit{cards}. Uma vez que cada \textit{board} corresponde a um projeto, é possível criar uma \textit{organization} que será o agrupamento de \textit{boards}.

\end{glossary}