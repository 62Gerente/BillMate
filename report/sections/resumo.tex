\begin{resumo}

Os projetos devem ser realizados com base em ideias novas, e não se devem focar em reinventar a roda. Foi este objetivo que moveu o grupo, e que desde o inicio pensou em realizar algo diferente e que fizesse a diferença no dia-a-dia do público-alvo.

Desde cedo surgiram várias ideias, e todas elas tinham bastante potencial, porém, o grupo queria algo que fosse útil para uso próprio e para aquelas que nos rodeiam. Assim, idealizou-se uma aplicação que pudesse ser realizada em contexto académico mas que no final da sua implementação pudesse ficar online para todos utilizarem. Com isto, pensou-se sobre uma aplicação que ajudasse na divisão de despesas entre um grupo de pessoas.

Dito isto, iniciou-se uma pesquisa que permitiu concluir que já existem algumas com o mesmo objetivo, mas, constatou-se que todas elas tentam ser muito genéricas, perdendo alguma abstração aquando da sua integração com diferentes tipos de utilizadores, e no caso dos estudantes, o grupo verificou que as existentes não abordam conceitos como "pagamento da eletricidade", ou um simples "pagamento da internet".

Com base nesta investigação, focou-se o público-alvo para estudantes, e inseriram-se todos os conceitos importantes para uma gestão organizada das despesas dos mesmos.

Todo o trabalho que se documenta neste relatório, descreve todo o processo de criação de uma aplicação que suporta a divisão de despesas de estudantes, ajudando a fazer a divisão entre os diversos grupos em que se encontra, para que este saiba em qualquer momento, as despesas que já estão pagas, as datas limite de pagamento, entre vários outros componentes que se referirão nos capítulos posteriores.

\end{resumo}