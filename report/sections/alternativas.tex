\section{Alternativas}

De seguida serão apresentadas algumas soluções que apesar de não serem exatamente iguais à que se está a representar neste documento, apresentam algumas características comuns. Assim, serão apresentadas algumas vantagens e desvantagens.

\subsection{Documento de Texto}

Esta forma de divisão de contas, é o mais básico que pode ser realizado, contudo é muito desorganizado e certamente centralizado. Assim, o utilizador regista de forma textual todas as suas dívidas, assim como todas as quantias que têm a receber. Esta até pode ser uma boa solução, se a cardinalidade dos registos for baixa, contudo, quando o volume de dados aumento, a complexidade na gestão da informação também será muito maior.

Além disto, podem haver muitos problemas com valores inconsistentes, porque como não há sincronização entre os utilizadores, estes podem colocar valores distintos nos seus registos.

Importa ainda salientar, que com esta abordagem, toda a consulta e gestão de dívida é muito mais complicada, porque o processo não é automático.

Este formato é completamente desaconselhado, mas acredita-se que ainda há muitos grupos de pessoas que preferem utilizar este tipo de organização.

\subsection{Folhas de Cálculo}

Contrariamente à aboragem apresentada na subsecção anterior, a utilização de folhas de cálculo (ficheiros Excel, etc) é uma melhor alternativa. Contudo, o problema da sincronização continua presente. É certo que com as tecnologias habituais, já é possível sincronizar os mesmos, porém o processo de partilha era mais complicado, uma vez que seria necessário partilhar o documento apenas com as pessoas que lhe dizem respeito, tendo no limite, um ficheiro para cada evento.

Além disso, a consulta de valores necessários para a tomada de decisão é mais delicada, sendo necessário analisar todos os ficheiros e agregar apenas os valores oportunos.

A vantagem da utilização desta abordagem deve-se ao facto de que cada utilizador cria o ficheiro de acordo com as suas necessidades, ajustando-o aos seus gostos. Mas, é óbvio que este requer algum tempo para ser construído, obrigando a que o utilizador tenha algum conhecimento para que as possa ajustar às suas necessidades. Além disso, é certo que estas estarão mais suscetíveis a erros.

\subsection{Aplicações}

Além das alternativas que foram referidas, uma outra alternativa é a utilização de aplicações semelhantes à que está a ser desenvolvida. Todas as aplicações, mesmo que tenham o mesmo tema, são sempre realizadas de forma diferente. Variam essencialmente, nas suas funcionalidades, nos seus algoritmos, na sua simplicidade, e obviamente na interface gráfica.

No contexto do projeto em desenvolvimento, tal como já foi referido, a maioria das aplicações existentes tentam ser genéricas para poderem obter um maior número de utilizadores, e com isto reparou-se que estas falham em alguns aspetos, tal como a abstração que se tem perante um tema. A título de exemplo, para um casal que habita numa casa própria, uma despesa de eletricidade pode ter uma semântica, mas para um estudante que partilha a casa com mais colegas terá outro significado.

O aumento da abstração dos conceitos é o principal problema com que o grupo se deparou, porque para se conseguir abranger todos os tipos de grupos, é necessário ter conhecimento de todos os estilos de pessoas que irão utilizar a aplicação, e isso será um pouco complicado.
Deste modo, serão apresentadas algumas alternativas, que apesar de não serem totalmente orientadas para os estudantes, permitem que possam ser usadas pelos mesmos.

\subsubsection{Split Wise}
\mbox{}\\
\-\hspace{0.3cm} O \textit{Split Wise} é uma solução para quem pretende dividir despesas de habitação ou ainda despesas de viagens.
Com esta apenas é possível adicionar as despesas e os intervenientes nessa despesa, e de seguida o valor é partilhado entre todos de forma equitativa.

Este tem ainda a opção de definir prazos para pagamento, que faz com que o utilizador seja alertado aquando da aproximação da data de pagamento.

\subsubsection{Split a Bill}
\mbox{}\\
\-\hspace{0.3cm} O \textit{Split a Bill} permite a partilha de despesas de qualquer natureza. Este é muito abstrato nos seus conceitos, não aplicando semântica aos termos.

Nesta aplicação, uma despesa tem apenas um nome e uma descrição. Aqui, não há tipos de despesas, nem nenhuma diferenciação de conceitos.

Apenas possui funcionalides de criar uma despesa, associar pessoas e enviar-lhes mensagens com os valores em débito.

\subsubsection{We Split}
\mbox{}\\
\-\hspace{0.3cm} O \textit{We Split} e muito semelhante ao \textit{Split a Bill}, uma vez que se abstrai dos conceitos, tentando ser genérico.

As funcionalides existentes também são muito semelhantes, mas este introduz a noção de grupos, onde cada um pode ter associado diferentes dívidas e diferentes pessoas.

Contudo, a semântica introduzida nos conceitos continua a ser desconhecida, não distinguindo os tipos envolvidos, que serão úteis para questões de análises mais aprofundadas.

