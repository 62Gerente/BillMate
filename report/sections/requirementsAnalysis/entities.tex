\section{Entidades}

Após uma análise aos objetivos do sistema \textit{BillMate}, definiram-se como entidades, as que se encontram na lista seguinte. Essas entidades que são listadas são as candidatas a serem classes aquando da modelação do sistema.

\begin{itemize}
\item Acção - Corresponde ao histórico de operações. É gerada uma acção quando é executada qualquer operação no sistema. Todas as notificações serão geradas através das acções.
\item Casa - Tipo de grupo que o utilizador pode criar e tem características específicas para uma casa.
\item Coletivo - Tipo de grupo que o utilizador pode criar e tem características mais genéricas que podem ser usadas para outros tipos de círculos.
\item Tipo de despesa - Define o tipo de despesa que vai ser utilizado para as despesas.
\item Dívida - Indica o valor que cada utilizador deve em cada despesa.
\item Débito Direto - Entidade que permite criar débitos diretos, para tornar o processo de criação de despesas automático.
\item Despesa - Identifica as despesas que os utilizadores têm para os vários grupos
\item Pagamento - Entidade que representa o pagamento de um utilizador para uma determinada despesa.
\item Utilizador referenciado - Utilizador que não está registado mas que pode ser utilizado no processo de divisão de despesas.
\item Utilizador registado - Utilizador que está devidamente registado no sistema.
\item Despesa regular - Despesa que é regular de acordo com uma periodicidade. Permite relembrar o utilizador que tem de criar a despesa e permite que seja criada mais rapidamente.
\item Subscrição - Utilizador subscreve-se para receber novidades do sistema.
\item Notificação do sistema - Aviso que é apresentado ao utilizador sobre acções que lhe estão associadas.
\end{itemize}