\chapter[Conclusão e trabalho futuro]
{Conclus\~ao e trabalho futuro}

Após o término do trabalho, é possível denotar alguns aspetos que não correram da melhor maneira, destacando sobretudo o facto do desconhecimento da framework. Com a evolução da implementação, foi se conhecendo novas maneiras de implementar, cada vez mais eficientes, que se fossem desenvolvidas desde o início, poderia se otimizar muito do código implementado.

Deve-se salientar também que a modelação deveria ter sido realizada com mais detalhe, porque a certa altura, teve-se de alterar os domínios do sistema para tornar a aplicação mais eficiente. Esta alteração obrigou a que fosse atualizada bastante da lógica de negócio.
A infraestrutura foi realizada com sucesso, e conseguiu-se ultrapassar os objetivos que foram propostos, mas tem-se consciência que se tivesse havido mais algum estudo sobre os recursos que foram utilizados, conseguiria-se ter menos máquinas a fazer a mesma coisa.

Um ponto negativo que também importa referir, é que os cenários de teste não foram realizados com o rigor que se pretendia, porque esse aspeto é um ponto que deve ser muito bem planeado, e nos quais não foram bem identificados.

Contudo, este trabalho permitiu melhorar algumas competências em programação orientada a objetos, perceber melhor aspetos ligados a \textit{enterprise application} que até este momento eram desconhecidos pela maioria dos elementos do grupo e começar a lidar com problemas reais como testes e \textit{deploy} da aplicação.

Como trabalho futuro, pretende-se melhorar a aplicação, otimizando e fazendo mais testes de integração. Pretende-se ainda implementar funcionalidades adicionais, tal como o triangulação de despesas que foi proposto mas não foi realizado. Terminar a aplicação móvel e lançamento da versão beta.