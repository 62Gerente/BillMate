\chapter[Introdução]
{Introdu\c{c}\~ao}


O trabalho que se encontra documentado neste relatório é relativo ao Projeto Integrado enquadrado na UCE de Engenharia de Aplicações, sendo que o objetivo deste projeto é conseguir utilizar os conceitos abordados nos quatro módulos desta UCE para se conseguir construir uma plataforma que suporte a aplicação com alta disponibilidade e com segurança.

Com base neste objetivo, delineou-se o desenvolvimento de uma aplicação de despesas partilhadas entre estudantes. Esta ideia surge porque, a maioria dos jovens começa a ter mais necessidades de organização aquando da sua chegada à universidade, que é quando normalmente sai da casa dos pais, e precisa de começar a pagar as suas obrigações, sejam elas de eletricidade, de gás, entre outras. Com base nisto, é certo que por várias vezes, as dívidas caem em esquecimento, e muitas são as histórias de terem ficado sem eletricidade, sem gás ou sem internet por falta de pagamento.

O nome \emph{BillMate} é uma concatenação de \emph{Bill} com \emph{Mate}, que significa exatamente aquilo que esta aplicação faz, que é a partilha de despesas entre colegas. Esta é uma excelente ferramenta para os estudantes, porque é feita por estudantes, que sabem exatamente as suas necessidades. É certo que não há melhor cliente do que o próprio cliente, uma vez que conhece todas as suas necessidades. Além disso, a facilidade de confrontar o público-alvo com esta aplicação, irá permitir que esta seja mais cómoda para o utilizador final.

O trabalho que se documenta neste relatório, descreve todo o processo de criação de uma aplicação que suporta a divisão de despesas de estudantes, ajudando a fazer a divisão entre os diversos grupos em que se encontra, para que este saiba em qualquer momento, as despesas que já estão pagas, as datas limite de pagamento, as próximas despesas, as pessoas que já pagaram, entre vários outros componentes que se referirão nos capítulos posteriores.

Os capítulos encontram-se organizados estrategicamente, apresentando-se Inicialmente uma motivação para a realização deste projeto, de modo que o utilizador final perceba quais os problemas que esta aplicação vem resolver. Esta conclusão do utilizador final terá mais ênfase quando confrontado com os casos de estudo que aparecerão posteriormente. Após estes, serão verificados os requisitos e o planeamento tomado em consideração. Com isto, entrar-se-à em detalhes mais técnicos, onde será descrita toda a infraestrutura que suportará a aplicação, passando pela modelação, interface e implementação da codificação do sistema.
Terminando todo este processo serão apresentados os resultados de desempenho da aplicação na infraestrutura desenvolvida.
