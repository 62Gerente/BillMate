\chapter[Modelação]
{Modela\c{c}\~ao}

Após o levantamento e a análise de requisitos, o processo de modelação torna-se mais simplificado. Obviamente que não é um processo simples, mas já se conhecem os requisitos que devem ser tomados em consideração na modelação.

Nas próximas subsecções, irão ser apresentados três diagramas que o grupo considerou ser mais relevantes para a realização do projeto. Inicialmente elaborou-se o diagrama de use case, onde se descreve as funcionalidades que são precisas. Após se ter o conhecimento das funcionalidades, definiu-se o relacionamento presente entre as várias entidades do sistema. Por fim, indicam-se as classes que são representadas no sistema.

De seguida apresenta-se de forma mais detalhada a modelação de cada um dos diagramas.

\section{Diagrama de Use Case}

O diagrama de \textit{use case} é feito para se perceber as funcionalidades do sistema. Este, representa a interação entre um utilizador e o sistema. Com a elaboração destes, conseguiu-se perceber a unidade de um trabalho significante. Cada caso dos que serão apresentados na imagem seguinte, descreve a funcionalidade que irá ser construída no sistema proposto.

Importa salientar que com este diagrama, não se pretende que definir como o software deverá ser construído, mas sim como se deve comportar quando estiver pronto.

O desenvolvimento de um \textit{software} é algo bastante complexo, e o desenvolvimento dos diagramas de \textit{use case} descrevem uma "fatia" do que o \textit{software} deverá oferecer.

Estes, serão também os mais indicados para o cliente final visualizar, porque são construídos com linguagem natural, facilmente percetível por qualquer pessoa. \\


\begin{figure}[H]
\centerline{\includegraphics[width=1\textwidth]{images/modeling/useCase}}
\caption{Diagrama de use case}
\end{figure}

\section{Diagrama de Modelo de Domínio}

Tal como o próprio nome indica, domínio, é utilizado para denotar áreas funcionais dentro de sistemas que exibem funcionalidades similares. Este diagrama pode ser interpretado como sendo uma coleção de componentes de software que partilham um determinado conjunto de características.

O objetivo desta análise deve-se ao facto de se puder analisar a informação que é identificada, capturada e organizada, para que se possa reutilizar na interação entre os domínios. É certo que esta reutilização está a ser vista a um nível de abstração muito elevado, uma vez que neste momento apenas se está a analisar o domínio, mas é útil aquando da construção do diagrama de classes. Apesar de não ser este o objetivo, esta modelação será útil se for necessário que as funcionalidades sejam reutilizadas para múltiplos sistemas. \\

\begin{figure}[ht]
\centerline{\includegraphics[width=1\textwidth]{images/modeling/modeloDominio}}
\caption{Diagrama do modelo de domínio}
\label{fig:domainModel}
\end{figure}

Tal como se pode analisar pela imagem \ref{fig:domainModel}, um utilizador é uma das entidades principais do sistema, uma vez que é este que despoleta as ações. Este utilizador pode ser classificado como registado ou referenciado. Este distinção deve-se ao facto de um utilizador não ser registado e puder ser utilizado na aplicação para partilhar despesas.
Cada utilizador encontra-se em um ou vários círculos, sendo que um círculo pode ser classificado como um tipo específico (casa), e um tipo mais genérico (colectivos).
Um determinado utilizador que se encontra em um determinado círculo tem despesas, que são partilhadas com os restantes utilizadores daquele mesmo círculo.
Definiu-se o tipo despesa no modelo de domínio, de modo a agrupar as despesas por categorias, sendo elas por exemplo, de eletricidade, de gás, entre outras, que podem até ser personalizadas.
As despesas regulares são criadas para alertar os utilizadores quando se aproxima a data de receção da fatura. Esta data, terá de ser obviamente definida pelo utilizador, que além desta data define a periodicidade com que esta se repete, normalmente mensal, mas é personalizável.
Cada despesa pode ter uma fatura e um recibo, assim como um débito direto.
Todas as ações que são feitas pelos utilizadores, geram notificações para darem feedback constante ao utilizador.
\section{Diagrama de Classes}

O diagrama de classes é uma das peças de modelação que mais fazem sentido utilizar, porque é uma representação da estrutura e das relações das classes que servem de modelo para objetos.

Estes definem todas as classes que o sistema necessita de ter e é a base para a construção dos restantes diagramas de comunicação, sequência e de estados.
Com a utilização deste diagrama é possível visualizar a representação da estrutura do sistema recorrendo ao conceito de classes e relações. Este modelo resulta de um processo de seleção onde são identificados os objetos relevantes do sistema em estudo e que se pretende descrever no seu ambiente.

Optou-se pela utilização deste diagrama, pois deste modo consegue-se visualizar como cada classe se relaciona com as restantes, tendo como objetivo a satisfação dos requisitos funcionais definidos para o sistema em estudo. A legibilidade do mesmo permite que a transição para implementação de código seja facilmente interpretada.

\begin{figure}[H]
\centerline{\includegraphics[width=1\textwidth]{images/modeling/diagramaClasses}}
\caption{Diagrama do modelo de domínio}
\label{fig:classDiagram}
\end{figure}

Inicialmente optou-se pela utilização de herança, contudo, quando se estava a fazer a modelação reparou-se que não estava a ser feita a correta reutilização dos recursos, sendo necessário instanciar o objeto para o poder reutilizar. Um exemplo simples é o caso do utilizador referenciado. Supondo que existe um utilizador que é referenciado com o email abc@abc.com. Se for implementada a herança, quando ele se regista é preciso instanciar um utilizador registado. Com composição, pode-se utilizar o \textit{design pattern state}, que faz com que não seja necessário uma nova instância, mudando apenas o estado de utilizador registado para utilizador referenciado.

Normalmente, deve-se preferir a composição sobre a herança, mas obviamente que existem exceções. Basicamente, usa-se a herança quando se sabe que a superclasse não vai variar, porque caso contrário será necessário alterar todas as classes que a implementam. Neste caso, está-se claramente a ver que a herança era a pior escolha em todos os casos, porque, quer-se um relacionamento do tipo "tem um", por exemplo, o utilizador registado tem um utilizador.

De seguida serão apresentadas todas as classes do sistema de uma forma mais detalhada, explicando os seus relacionamentos e a sua cardinalidade.
\begin{itemize}
	\item \textbf{\textit{User}}\\
	O \textit{registered user} e o \textit{referred user} têm um \textit{user}. Aqui tem-se a composição de ambos devido ao que foi explicado na introdução desta secção.
	Sendo o objetivo desta aplicação a partilha de despesas entre várias pessoas, então, a classe \textit{user} tem uma \textit{debt} que a relaciona com a classe \textit{expense}. Esta classe tem ainda outro relacionamento com a classe \textit{payment} que o relaciona com a classe \textit{debt}. Isto ocorre porque um utilizador pode fazer vários pagamentos para pagar uma despesa.

	\item \textbf{\textit{Circle}}\\
	A classe \textit{house} e \textit{collective} têm um \textit{circle}. Com isto, verifica-se que um \textit{circle} tem um ou mais \textit{users}, mas os \textit{users} podem não estar em nenhum \textit{circle}.
	O \textit{circle} tem um ou mais \textit{expenseTypes}, que serão utilizados para criar as despesas para um determinado círculo. Esta classe contém \textit{expenses} efetuadas por membros do círculo.

	\item \textbf{\textit{Expense}}\\
	Esta é a classe mais importante do sistema, uma vez que se relaciona com todas as classes. Este é bastante significativo porque indica os utilizadores que têm pagamentos em dívidas realizadas num determinado círculo.

	\item \textbf{\textit{Payment}}\\
	Os \textit{payments} indicam os utilizadores que têm pagamentos numa determinada dívida. Um \textit{user} pode ter vários \textit{payments}, porque pode realizar vários pagamentos para pagar uma dívida.

	\item \textbf{\textit{Debt}}\\
	Como já deu para perceber anteriormente, a classe \textit{Debt} indica a dívida que um utilizador tem numa determinada despesa. Um \textit{user} pode ter várias \textit{debts}, mas cada \textit{debt} pertence a um \textit{user}. Do mesmo modo, uma \textit{expense} pode ter zero ou várias \textit{debts}, mas cada \textit{debt} é relativa apenas a uma \textit{expense}.

	\item \textbf{\textit{Action}}\\
	A classe \textit{action} é gerada sempre que qualquer utilizador realizar uma determinada tarefa no sistema.

	\item \textbf{\textit{Notification}}\\
	As notificações serão geradas a partir das \textit{actions} realizadas pelos \textit{users}.

	\item \textbf{\textit{Regular Expense}}\\
	Esta classe é criada sempre que um utilizador quiser criar uma despesa regular, que basicamente o alerta sobre as próximas faturas, para que este não se esqueça de criar as despesas. Com isto, é possível criar uma despesa sem esforço para o utilizador, porque os atributos da despesa são preenchidos com os atributos da despesa regular.

	\item \textbf{\textit{Expense Type}}\\
	Esta classe tem os tipos de despesa que são apresentados no momento em que se criam despesas, porque é obrigatório que cada despesa tenha um tipo de despesa para se efetuar uma organização de despesas mais eficaz na lógica de negócio.

	\item \textbf{\textit{Circle Type}}\\
	Esta classe armazena os tipos de despesa que são apresentados no círculo no momento da sua criação. As que aparecem ao utilizador são as \textit{default expense types}, mas caso queira criar uma nova será adicionada ao \textit{custom expense type}.

\end{itemize}

